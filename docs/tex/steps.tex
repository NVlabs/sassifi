\section{Getting started with SASSIFI}

\subsection{Prerequisites}

\begin{itemize}

\item A linux-based system with an x86 64-bit host, a Fermi-, Kepler-, or
Maxwell-based GPU. SASSIFI has been tested on Ubuntu (12) and CentOS (6).

\item Python 2.7 is needed to run the scripts provided to generate injection
sites, launch injection campaign, and parse the results.	

\begin{itemize}
\item The lockfile module is needed to run the injection jobs in parallel
either on a multi-gpu system or a cluster of nodes with shared filesystem. 
Instructions to install the lockfile module can be found here~\cite{lockfile}.
This module is not needed if you want to run injection jobs sequentially on a
single-gpu system. 
\item (Optional) The final results can be parsed into an xlsx file using the
xlsxwriter module. Instructions to install xlswriter can be found
here~\cite{XlsxWriter}. The results will be parsed into multiple text files, if
you do not have xlswriter.  These can be copied into an excel file to plot and
visualize the results.
\end{itemize}

\item SASSI, which can be downloaded from GitHub~\cite{SASSI}. SASSIFI is
tested using the latest commit (5523d984ad047a272297c1a3ff8c63f55c0ad026).
SASSIFI provides code  that needs to be compiled using the SASSI
framework.  This code includes SASSI handlers that execute code before and
after instructions for profiling and error injections. Please follow the steps
provided in the SASSI documentation to install SASSI. 

\end{itemize}


% Directory structure 
\subsection{SASSIFI package directory structure}

\dirtree{%
.1 SASSIFI\_HOME.
%
.2 err\_injector \ldots{} \begin{minipage}[t]{12cm} 
					Source code of the SASSI handlers for application profiling and error injection.\ This directory should be moved to the SASSI source directory.\
					\end{minipage}.
.3 error\_injector.h.
.3 injector.cu.
.3 profiler.cu.
.3 Makefile.
.3 copy\_handler.sh. 
%
.2 scripts \ldots{} \begin{minipage}[t]{12cm} 
					Scripts to generate injection list, run injections, and parse results.\
					\end{minipage}.  
.3 common\_params.py.
.3 specific\_params.py.
.3 common\_functions.py.
.3 generate\_injection\_list.py.
.3 run\_one\_injection.py.
.3 run\_injections.py.
.3 parse\_results.py.
.3 process\_kernel\_regcount.py.
%
.2 suites \ldots{} \begin{minipage}[t]{12cm} Workloads will be stored here.\  \end{minipage}.
.3 example \ldots{} \begin{minipage}[t]{12cm} We provide a sample benchmark suite named {\it example}.\  \end{minipage}.
.4 simple\_add \ldots{} \begin{minipage}[t]{12cm} Look at the makefile here.\  \end{minipage}.
.5 simple\_add.cu.
.5 Makefile.
%
.2 run \ldots{} \begin{minipage}[t]{12cm} 
				Stores run and sdc\_check scripts for different applications.\ The subdirectory structure here is similar to the suites directory above.\
				\end{minipage}.  
.3 example.
.4 simple\_add.
.5 sassifi\_run.sh.
.5 sdc\_check.sh.
%
.2 {\it bin} \ldots{} \begin{minipage}[t]{12cm} Stores application binaries.\ This directory will be auto generated by the makefile in the suites subdirectory.\  \end{minipage}.
.3 {\it none} \ldots{} \begin{minipage}[t]{12cm} Application binaries without instrumentation.\  \end{minipage}.
.3 {\it profiler} \ldots{} \begin{minipage}[t]{12cm} Application binaries instrumented for application profiling.\ \end{minipage}.
.3 {\it inst\_injector} \ldots{} \begin{minipage}[t]{12cm} Application binaries instrumented with instruction output-level error injection handler.\  \end{minipage}.
.3 {\it rf\_injector} \ldots{} \begin{minipage}[t]{12cm} Application binaries instrumented with register file level error injection handler.\  \end{minipage}.
% %
.2 {\it logs} \ldots{} \begin{minipage}[t]{12cm} Logs will be stored here after error injection runs.\ This directory will be auto generated by the scripts.\  \end{minipage}.
.3 {\it example} \ldots{} \begin{minipage}[t]{12cm} One directory per benchmark suite will be created.\  \end{minipage}.
.4 {\it simple\_add} \ldots{} \begin{minipage}[t]{12cm} One directory per workload.\ This directory will contain files that list the error injection sites and the results from the error injection campaigns.\  \end{minipage}.
.3 {\it results} \ldots{} \begin{minipage}[t]{12cm} Summary of the parsed results (excel sheet) will be stored here.\  \end{minipage}.
%	
.2 test.sh \ldots{} \begin{minipage}[t]{12cm} Sample script that automates several steps involved running SASSIFI.\ This is a great place to start, if you are short on time.\  \end{minipage}.
.2 docs.
.3 sassifi-user-guide.pdf \ldots{} \begin{minipage}[t]{12cm} This file!\ \end{minipage}.
.3 tex  \ldots{} \begin{minipage}[t]{12cm} LaTeX files used to generate this document.\ \end{minipage}.
}





\subsection{Setting up and running SASSIFI}

Follow these steps to setup and run SASSIFI. We provide a sample script (test.sh) that automates several of these steps.

\begin{enumerate}

\item {\bf Set the following environment variables}:
	\begin{itemize}
		\item SASSIFI\_HOME: Path to the SASSIFI package (e.g., /home/username/sassifi\_package/)
		\item SASSI\_SRC: Path to the SASSI source package (e.g., /home/username/sassi/)
		\item INST\_LIB\_DIR: Path to the SASSI libraries (e.g., SASSI\_SRC/instlibs/lib/)
		\item CCDIR: Path to the gcc version 4.8.4 or newer (e.g., /usr/local/gcc-4.8.4/)
		\item CUDA\_BASE\_DIR: Path to SASSI installation (e.g., /usr/local/sassi7/)
		\item LD\_LIBRARY\_PATH should include the cuda libraries (e.g., CUDA\_BASE\_DIR/lib64/  \\
		and CUDA\_BASE\_DIR/extras/CUPTI/lib64/)
		\item Ensure that the GENCODE variable is correctly set for the target GPU
		in SASSI\_SRC/instlibs/env.mk and application makefiles (e.g.,
		SASSIFI\_HOME/suites/example/simple\_add/Makefile. 
		\end{itemize}
\label{step1}

\item {\bf Copy the SASSI Fault Injection (SASSIFI) handler into the SASSI package}:
		 We provide err\_injector/copy\_handler.sh script to perform this step.
		 Simply run it from any directory. This script creates a new directory
		 named err\_injector in the SASSI\_SRC/instlibs/src directory and creates
		 soft-links for the files provided in the err\_injector directory to avoid
		 keeping multiple copies of the SASSI handler files.
\label{step2}

\item {\bf Compile the SASSIFI handlers}:
		Simply type {\em make} in \$SASSI\_SRC/instlibs/src/err\_injector.  This should
		create four libraries. The first one is for profiling the application and
		identifying how many injection points exist. The remaining three are for
		injecting errors during an application run (one each for the three 
		injection modes).
\label{step3}

\item {\bf Prepare applications}:
		\begin{enumerate}

			\item {\bf Record fault-free outputs:} Record golden output file (as
			golden.txt) and golden stdout (as golden\_stdout.txt) and golden stderr
			(as golden\_stderr.txt) in the workload directory (e.g.,
			\$SASSIFI\_HOME/suites/example/simple\_add/).
			\label{step41}

			\item {\bf Create application-specific scripts:} Create sassifi\_run.sh
			and sdc\_check.sh scripts in run/ directory.  These are workload specific
			and have to be manually created. Instead of using absolute paths, please
			use environment variables for paths such as BIN\_DIR, APP\_DIR,
			DATA\_SET\_DIR, and RUN\_SCRIPT\_DIR. These variables are set by
			run\_one\_injection.py script before launching error injections. See
			the bash scripts in the run/example/simple\_add/run/ directory for
			examples. You can also add an application specific check here. 
			\label{step42}

			\item {\bf Prepare applications to compile with the SASSIFI handlers}:
			This might require some work. Follow instructions in the SASSI
			documentation on how to compile your application with a SASSI handler.
			
			Tip: Prepare them such that you can type "make OPTION=profiler" to
			generate binaries to do the profiling step (step 4) and "make
			OPTION=inst\_value\_injector"  or "make OPTION=inst\_address\_injector" 
			or "make OPTION=rf\_injector" to generate
			binaries for error injection campaigns for the three injection modes (see
			Sections 1 and 2). See makefile in suites/example/simple\_add/ for an
			example. This makefile installs different versions of the binaries to 
			\$SASSIFI\_HOME/bin/\$OPTIONS/ directories. 
			\label{step43}

			\end{enumerate}

\item {\bf Profile the application}:
		Compile the application with "OPTION=profiler" and run it once with the
		same inputs that is specified in the sassifi\_run.sh script. A new file
		named sassifi-inst-counts.txt will be generated in the directory where the
		application was run. This file contains the instruction counts for all the
		instruction groups defined in err\_injector/error\_injector.h and all the
		opcodes defined in sassi-opcodes.h for all the CUDA kernels. One line is
		created per dynamic kernel invocation and the format in which the data is
		printed is shown in the first line in the sassifi-inst-counts.txt file. 
\label{step5}

\item {\bf Build the applications for error injection runs}:
		Simply run "make OPTION=inst\_value\_injector", "make OPTION=inst\_address\_injector" 
		and/or "make OPTION=rf\_injector"
\label{step6}

\item {\bf Generation injection sites}:
		\begin{enumerate}
			\item Ensure that the parameters are set correctly in specific\_params.py and
			common\_params.py files.  Some of the parameters that need user attention
			are: 
			\begin{itemize}
				\item Setting maximum number of error injections to perform per instruction
				group and bit-flip model combination. See NUM\_INJECTION and
				THRESHOLD\_JOBS in specific\_params.py file. 

				\item Selecting instruction groups and bit-flip models. See 
				rf\_bfm\_list,  inst\_value\_igid\_bfm\_map, and 
				inst\_address\_igid\_bfm\_map in specific\_params.py for the list of
				supported instruction groups (IGIDs) and bit-flip models (BFMs). Simply
				uncomment the lines to include the IGID and the associated BFMs. User
				can also select only a subset of the supported BFMs per IGID for
				targeted error injection studies.

				\item Listing the applications, benchmark suite name, application
				binary file name, and the expected runtime on the system where the
				injection job will be run. See the apps dictionary in
				specific\_params.py for an example. The dictionary and the strings
				defined here are used by other scripts to identify the directory
				structure in the suites directory and the application binary name.  The
				expected runtime defined here is used later to determine when to
				timeout injection runs (based on the TIMEOUT\_THRESHOLD defined in
				common\_params.py).

				\item Setting paths for the suites, logs, bin, and run directories if
				the user decides to use a different directory structure. If the
				directory structure for the new benchmark suite that you plan to use is different,
				please update the app\_dir[app] and app\_data\_dir[app] variables
				accordingly. 
			
				\item Setting the number of allocated registers per static kernel per
				application. When an application is compiled using {\it -Xptxas -v}
				flags, the number of registers allocated for each static kernel in the
				application are printed on the standard error (stderr).  User needs to
				parse the stderr and update the num\_regs dictionary in the
				specific\_params.py file. Obtain the number of allocated registers
				without SASSI instrumentation.  If num\_regs dictionary is incorrect
				(missing/extra kernel names, fewer/more registers per kernel), then the
				results will also be incorrect because the number of error injections
				are chosen based on num\_regs.  We provide the
				process\_kernel\_regcount.py script that parses the stderr from an
				input file and creates a dictionary per application which is stored in
				a pickle file. This pickle file can be loaded directly by the
				specific\_params.py (see set\_num\_regs() for an example). We process
				the stderr generated by compiling the simple\_add program using this
				script in test.sh.
				
				The num\_regs dictionary is needed for the RF mode injections. If you do
				not plan to perform RF mode injections, you can ignore this part.  

			
			\end{itemize}
			\label{step71}

			\item Run generate\_injection\_list.py script to generate a file that
			contains what errors to inject. Instructions are selected randomly for
			across the entire application for the RF mode and across the instructions
			from the specified instruction group in the IOV and IOA modes. Since we know the
			instruction count breakdown per kernel invocation from the profiling
			phase, we combine the instructions from all the kernel executions (based
			on the instruction groups) and randomly select dynamic instruction
			numbers for error injections. We map this dynamic instruction number back
			to a dynamic kernel invocation index, along with static kernel name. We
			create a random number (between 0 and 1) for selecting the destination
			register among the number of destination registers in the selected
			dynamic instruction for the IOV and IOA modes. We select the register number within
			the set of allocated registers for the selected static kernel for the
			RF mode. We also select an additional random number for selecting the
			bit location to inject the error (according to the chosen bit-flip model). 
			\label{step72}
		\end{enumerate}

\item {\bf Run injections}: 
		Run the run\_injections.py script to launch the error injection campaign.
		This script will run one injection after the other in the standalone mode.
		Please do not attempt to run multiple jobs in parallel unless you install
		the lockfile python module or modify the run\_one\_injection.py script such
		that it does not write to the same results file.  If you use the multigpu
		option, mutiple injection jobs will be launched in parallel depending on
		the number of GPUs present in the system and the parameter  specified in
		specific\_params.py (NUM\_GPUS).  If you have a cluster of nodes where you
		can launch injection jobs, you can write some code in the
		"check\_and\_submit\_cluster" function in run\_injections.py script to
		launch multiple jobs to the cluster.  

		Tip: Perform a few dummy injections before proceeding with full injection
		campaign. Go to step~\ref{step3} and look for DUMMY\_INJECTION flag in the
		makefile. Setting this flag will allow you to go through most of the SASSI
		handler code, but skip the error injection. This is to ensure that you are
		not seeing crashes/SDCs that you should not see.
\label{step8}

\item {\bf Parse results}:
		Use the sample parse\_results.py script to get an initial set of parsed
		results.  This script generates an excel workbook with three sheets, if the
		xlsxwriter python module is found in the system. If not, three text files
		are created. The first sheet/text file shows the fraction of executed
		instructions for different instruction groups and opcodes. The second
		sheet/text file shows the outcomes of the error injections.
		Table~\ref{tab:error-categories} explains how we categorize error outcomes.
		The third sheet/text file shows the average runtime for the injection runs
		for different applications, instruction groups, and bit-flip models. Based
		on how you want to visualize the results, you may want to modify the script
		or write your own. 
\label{step9}

\end{enumerate}


In the current setup,
steps~\ref{step1},~\ref{step2},~\ref{step3},~\ref{step42},~\ref{step43},
and~\ref{step71} have to be done manually.  Once this is done, the remaining
steps can be automated and we provide an example script (test.sh) to run these
steps using a single command (./test.sh from \$SASSIFI\_HOME). 

