\section{Error outcomes}

Table~\ref{tab:error-categories} shows how we categorize the outcomes of the error injection runs.

\begin{table}[tbp]
\caption{Error injection outcomes.}
\vspace*{-0.1in}
\label{tab:error-categories}
% \scriptsize 
\centering
\begin{tabular}{|l|p{3cm}|p{10cm}|}
\hline

	\multicolumn{1}{|c|}{Category} & \multicolumn{1}{c|}{Subcategory} & \multicolumn{1}{c|}{Explanation} \\
	\hline 
	\hline

	\multirow{4}{*}{Masked} & \multicolumn{2}{|p{13cm}|}{Application output is same as the error free output. No error symptom is observed.} \\ 
	\cline{2-3}
							& Value not read & Currently, this applies only to the RF mode injections. The register selected for error injection, but it was never read. \\
	\cline{2-3}
							& Written before being read & Currently, this applies only to the RF mode injections. The register selected for error injection was overwritten before being read. \\ 
	\cline{2-3}
							& Other reasons & For the RF mode injection, the injected error was consumed but masked later in the application. For the instruction output-level injections (IOV and IOA modes), this is the only the masked outcome subcategory. \\
	\hline
	\hline

	\multirow{2}{*}{DUE} & \multicolumn{2}{|p{13cm}|}{Executions that terminate early or hang.} \\ 
	\cline{2-3}
							& Timeout & Executions that do not terminate within an allocated threshold, which is configurable by changing TIMEOUT\_THRESHOLD in scripts/common\_params.py. Default is 10$\times$ the fault-free runtime. \\ 
	\cline{2-3}
							& Non zero exit status & Application exits with non-zero exit status. \\ 
	\hline
	\hline


	\multirow{8}{*}{Potential DUE} & \multicolumn{2}{|p{13cm}|}{Symptoms of an unsuccessful application run can be seen in either {\it stdout}, {\it stderr}, or kernel exit status. Executions with failure symptoms can be categorized as DUEs if the system has appropriate error monitors.} \\
	\cline{2-3}
							& Kernel error, but masked & One of the kernels did not complete successfully (detected by comparing kernel exit status with {\it cudaSuccess}). The output of the application, however, matches the fault-free output. \\ 
	\cline{2-3}
							& Kernel error, but SDC & One of the kernels did not complete successfully (detected by comparing kernel exit status with {\it cudaSuccess}). The output of the application does not match the fault-free output. \\ 
	\cline{2-3}
							& Recorded error messages in {\it stderr} & Error messages are recorded in the {\it stderr}. For applications that write to {\it stderr} in fault-free runs, the new {\it stderr} is different than the fault-free one. \\ 
	\cline{2-3}
							& Recorded error messages in {\it stdout} & Error messages are recorded in the {\it stdout}. \\
	\cline{2-3}
							& {\it dmesg} error and {\it stderr} file is different & Stderr is different, but messages are recorded in the linux kernel (accessed using {\it dmesg} utility). \\
	\cline{2-3}
							& {\it dmesg} error and stout file is different & {\it Stdout} is different, but messages are recorded in the linux kernel (accessed using {\it dmesg} utility). \\
	\cline{2-3}
							& {\it dmesg} error and the output file is different & Output file (if it exists for the application) is different, but messages are recorded in the linux kernel (accessed using {\it dmesg} utility).  \\
	\cline{2-3}
							& {\it dmesg} error and application specific check failed & User-specified application specific (SDC) check failed,  but messages are recorded in the linux kernel (accessed using {\it dmesg} utility). \\
	\hline
	\hline


	\multirow{3}{*}{SDC} & \multicolumn{2}{|p{13cm}|}{Application finishes without crashes, hangs, or failure symptoms but at least one of the outputs of the application is different.} \\ 
	\cline{2-3}
							& {\it Stdout} is different & Text printed in {\it stdout} is different. Output file generated by the application is identical to the fault-free run. \\ 
	\cline{2-3}
							&	Output is different &  The output file generated by the application is different than the output generated by the fault-free run. \\ 
	\cline{2-3}
							&	Application-specific check failed &  The application-specific check provided by the user failed. \\
	\hline 
	\hline
	
\end{tabular}
\end{table}
